\documentclass[main.tex]{subfiles}

\begin{document}

\chapter{Kasus}


\section{Nominativ}
\begin{itemize}
\item Naamwoordelijk gezegde na een koppelwerkwoord (zijn, schijn blijven, blijken, lijken, worden, heten).\\
z.B. Er gilt als einer der zichtigsten Politiker des Landes.
\item Anrede: Kommen Sie bitte herein, mein Lieber.
\end{itemize}
\section{Akusativ}
\subsection{Meewerkend voorwerp in accusatief:}
\begin{itemize}
\item fragen (aan iemand iets vragen)
\item bitten 
\item kosten (sich)
\item lehren (sich)
\item wissen lassen (laten weten)
\item lassen fühlen (laten voelen)
\end{itemize}

\begin{tabular}{ccc}
 & Steeds met accusatief &  \\ 
\hline
durch & $\leftrightarrow$ & door \\ 
für & $\leftrightarrow$ & voor \\ 
gegen & $\leftrightarrow$ & tegen \\ 
ohne & $\leftrightarrow$ & zonder \\ 
um & $\leftrightarrow$ & om (de hoek) \\ 
entlang & $\leftrightarrow$ & langs \\ 
\end{tabular} 

\subsection{Bijvoeglijk naamwoord}
\begin{itemize}
\item lang
\end{itemize}
\todo[inline]{klopt dit?}
\section{Dativ}
\subsection{Lijdend voorwerp in de datief:}
\begin{itemize}
\item schaden
\item helfen
\item gratulieren
\end{itemize}

\begin{tabular}{ccc}
 & Steeds met datief &  \\ 
\hline
aus & $\leftrightarrow$ & uit \\ 
bei & $\leftrightarrow$ & bij \\ 
mit & $\leftrightarrow$ & met \\ 
nach & $\leftrightarrow$ & na \\ 
von & $\leftrightarrow$ & van \\ 
zu & $\leftrightarrow$ & tegen \\ 
seit & $\leftrightarrow$ & sedert \\ 
gegenüber & $\leftrightarrow$ & tegenover \\ 
\end{tabular} 

\subsection{Bijvoeglijk naamwoord}

\begin{itemize}
\item dankbar
\item bewusst
\end{itemize}

\section{Genitiv}
\subsection{Lijdend voorwerp in de genitief:}
\begin{itemize}
\item gedenken
\item bedürfen (nodig hebben)
\item enthalten (sich, zich onthouden)
\end{itemize}

\subsection{Von +datief of gentief?}
Voorwaarden:
\begin{enumerate}
\item het bezit (substantief) staat vooraan
\item de genitief moet realiseerbaar zijn
\todo[inline]{wat wordt bedoelt met realiseerbaar zijn?}
\item er is sparke van bezit
\end{enumerate}
Indien één van deze voorwaarden niet voldaan is dan wordt de genitief vertaalt met von + datief.
\begin{tabular}{ccc}
 & Steeds met genitief &  \\ 
 \hline
statt & $\leftrightarrow$ & in de plaats (van) \\ 
trotz & $\leftrightarrow$ & ondanks \\ 
während & $\leftrightarrow$ & tijdens \\ 
wegen & $\leftrightarrow$ & wegens \\ 
laut & $\leftrightarrow$ & volgens \\ 
dank & $\leftrightarrow$ & dankzij \\ 
\end{tabular} 

\section{Wisselvoorzetsels}

\begin{tabular}{ccc}
 & Wisselvoorzetsels &  \\ 
an & $\leftrightarrow$ & aan \\ 
auf & $\leftrightarrow$ & op \\ 
hinter & $\leftrightarrow$ & achter \\ 
neben & $\leftrightarrow$ & naast \\ 
in & $\leftrightarrow$ & in \\ 
zwischen & $\leftrightarrow$ & tussen \\ 
über & $\leftrightarrow$ & over \\ 
vor (plaats) & $\leftrightarrow$ & voor \\ 
unter & $\leftrightarrow$ & onder \\ 
\end{tabular} 

\subsection{Algemene regel}
\subsubsection{Lokativ}
ich laufe in den Wald: + Bewegung + Grenzüberschreitung $\rightarrow$ A\\
ich laufe in dem Wald: + Bewegung - Grenzüberschreitung $\rightarrow$ D\\
ich schlafe in dem Wald : -Bewegung (- Grenz"uberschreitung) $\rightarrow$ D
\subsubsection{nichtlokativ}
auf/über $\rightarrow$ A\\
der Rest $\rightarrow$ D
\subsubsection{temporal}
Standaard in \textbf{datief}\\


\subsection{Naar in het Duits}
\begin{tabular}{| c | c | c | c |}
\hline
\multicolumn{4}{|c|}{\textbf{keine Bewegungskontext}}\\
\hline
\multicolumn{4}{|c|}{nach +dativ}\\
\hline
\hline
\multicolumn{4}{|c|}{\textbf{Bewegungskontext}}\\
\hline
nach & \multicolumn{3}{|c|}{Stadt/Land Ohne Artikel}\\
\hline
in + A & \multicolumn{3}{|c|}{Lokalität MIT Artikel (altijd grensoverschreiding ook)}\\
\hline
zu + D & \multicolumn{3}{|c|}{in allen anderen Fällen}\\
\hline
\end{tabular}
\end{document}
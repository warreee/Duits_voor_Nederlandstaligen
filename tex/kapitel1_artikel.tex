\documentclass[main.tex]{subfiles}

\begin{document}

\chapter{Der Artikel}
\section{Het bepaald lidwoord}
\subsection{De standaardvervoeging}
\begin{tabular}{|c|c|c|c|c|}
\hline 
\rowcolor{gray}
 & männlich & weiblich & sächlich & Plural \\ 
\hline 
\cellcolor[gray]{0.8}Nominativ & der & die & das & die \\ 
\hline 
\cellcolor[gray]{0.8}Akkusativ & den & die & das & die \\ 
\hline 
\cellcolor[gray]{0.8}Dativ & dem & der & dem & den \\ 
\hline 
\cellcolor[gray]{0.8}Genitiv & des & der & des & der \\ 
\hline 
\end{tabular} 
\subsubsection{De samentrekking met het bepaald lidwoord}
In sommige gevallen wordt een vervoeging van het bepaald lidwoord samengetrokken met een voorzetsel:
\\
\\
\begin{tabular}{ccccc}
an, bei, in, von, zu & + & dem & $\rightarrow$ & am, beim, im, vom, zum \\ 
\hline 
zu & + & der & $\rightarrow$ & zur \\ 
\hline 
an, auf, in, um & + & das & $\rightarrow$ & ans, aufs, ins, ums \\ 
\end{tabular} 
\subsection{Hetzelfde vervoegd}

De volgende woorden worden vervoegd zoals de standaardvervoeging van het bepalend lidwoord. Het enige verschil is in de nominatief en accusatief onzijdig, hier wordt dan +ES gedaan ipv. +S.

\begin{tabular}{|c|c|c|c|}
\hline 
\rowcolor{gray}
Nederlands & Duits & enkelvoud & meervoud \\ 
\hline 
dit & dies- & x & x \\ 
\hline 
dat & jen- & x & x \\ 
\hline 
meerdere, sommigen & manch- & x & x \\ 
\hline 
zo'n, zulk & solch- & x & x \\ 
\hline 
welk & welch- & x & x \\ 
\hline 
al, (compleet) & all- & x & x \\ 
\hline 
iedereen, elk, allen & jed & x &  \\ 
\hline 
beide, allebei & beid- &  & x \\ 
\hline 
alle, hele & sämtlich &  & x \\ 
\hline 
\end{tabular} 
\section{Het onbepaald lidwoord}
\subsection{De standaardvervoeging}

\begin{tabular}{|c|c|c|c|c|}
\hline 
\rowcolor{gray}
& männlich & weiblich & sächlich & plural \\ 
\hline 
\cellcolor[gray]{0.8}Nominativ & \textcolor{red}{ein} & eine & \textcolor{red}{ein} & keine \\ 
\hline 
\cellcolor[gray]{0.8}Akkusativ & einen & eine & \textcolor{red}{ein} & keine \\ 
\hline 
\cellcolor[gray]{0.8}Dativ & einem & einer & einem & keinen \\ 
\hline 
\cellcolor[gray]{0.8}Genitiv & eines & einer & eines & keiner \\ 
\hline 
\end{tabular} 

\subsection{Hetzelfde vervoegd}
De volgende woorden worden vervoegd zoals de standaardvervoeging van het onbepaald lidwoord.

\begin{tabular}{|c|c|}
\hline 
\rowcolor{gray}
niederländisch & deutsch \\ 
\hline 
kein & geen \\ 
\hline 
mein & mijn \\ 
\hline 
dein & jouw \\ 
\hline 
sein & zijn \\ 
\hline 
ihr & haar \\ 
\hline 
unser & ons \\ 
\hline 
euer & jullie \\ 
\hline 
ihr & hun \\ 
\hline 
Ihr & uw \\ 
\hline 
was für ein? & wat voor een? \\ 
\hline 
\end{tabular} 

\section{Het onbepaalde lidwoord zelfstandig gebruikt}
Net zoals in het Nederlands kan het onbepaald lidwoord zelfstandig gebruikt worden:
\\
\\
\begin{tabular}{|c|c|c|c|c|}
\hline 
\rowcolor{gray}
& männlich & weiblich & sächlich & plural \\ 
\hline 
\cellcolor[gray]{0.8}Nominativ & \textcolor{red}{einer} & eine & \textcolor{red}{eines} & keine \\ 
\hline 
\cellcolor[gray]{0.8}Akkusativ & einen & eine & \textcolor{red}{eines} & keine \\ 
\hline 
\cellcolor[gray]{0.8}Dativ & einem & einer & einem & keinen \\ 
\hline 
\cellcolor[gray]{0.8}Genitiv & eines & einer & eines & keiner \\ 
\hline 
\end{tabular} 

\section{Samenvatting}
Voor de vervoeging van de lidwoorden kunnen we volgned stramien volgen:
\begin{enumerate}
\item Zelfstandig gebruik?
\item Bepaald of onbepaald lidwoord?
\item Een woord uit de tabel?
\item Als laatste criterium
	\begin{itemize}
	\item Geslacht?
	\item Enkelvoud of meervoud?
	\item Naamval
	\end{itemize}
\end{enumerate}
\end{document}

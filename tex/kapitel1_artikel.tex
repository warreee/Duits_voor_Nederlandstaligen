\documentclass[main.tex]{subfiles}

\begin{document}

\chapter{Artikel}
\section{Het bepaald lidwoord}
\subsection{De standaardvervoeging}
\begin{tabular}{|c|c|c|c|c|}
\hline 
\rowcolor{gray}
 & männlich & weiblich & sächlich & Plural \\ 
\hline 
\cellcolor[gray]{0.8}Nominativ & der & die & das & die \\ 
\hline 
\cellcolor[gray]{0.8}Akkusativ & den & die & das & die \\ 
\hline 
\cellcolor[gray]{0.8}Dativ & dem & der & dem & den \\ 
\hline 
\cellcolor[gray]{0.8}Genitiv & des & der & des & der \\ 
\hline 
\end{tabular} 

\subsection{Andere bepalende woorden}

De volgende woorden worden vervoegd zoals de standaardvervoeging van het bepalend lidwoord. Het enige verschil is in de 

\begin{tabular}{|c|c|c|c|}
\hline 
\rowcolor{gray}
Nederlands & Duits & enkelvoud & meervoud \\ 
\hline 
dit & dies- & x & x \\ 
\hline 
dat & jen- & x & x \\ 
\hline 
meerdere,sommigen & manch- & x & x \\ 
\hline 
zo'n, zulk & solch- & x & x \\ 
\hline 
welk & welch- & x & x \\ 
\hline 
al, (compleet) & all- & x & x \\ 
\hline 
iedereen, elk, allen & jed & x &  \\ 
\hline 
beide, allebei & beid- &  & x \\ 
\hline 
alle, hele & sämtlich &  & x \\ 
\hline 
\end{tabular} 
\section{Het onbepaald lidwoord}


\begin{tabular}{|c|c|c|c|c|}
\hline 
\rowcolor{gray}
& männlich & weiblich & sächlich & plural \\ 
\hline 
\cellcolor[gray]{0.8}Nominativ & \textcolor{red}{ein} & eine & \textcolor{red}{ein} & keine \\ 
\hline 
\cellcolor[gray]{0.8}Akkusativ & einen & eine & \textcolor{red}{ein} & keine \\ 
\hline 
\cellcolor[gray]{0.8}Dativ & einem & einer & einem & keinen \\ 
\hline 
\cellcolor[gray]{0.8}Genitiv & eines & einer & eines & keiner \\ 
\hline 
\end{tabular} 

\end{document}

\documentclass[main.tex]{subfiles}

\begin{document}

\chapter{Das Substantiv}
\section{Das Genus}
De volgende secties geven aanwijzigingen om het geslacht van het substantief te bepalen.
\subsection{Vorm}
\begin{minipage}[t]{0,33\textwidth}
 \subsubsection{Männlich}
 \begin{itemize}
 \item -ling
 \item -ig
 \item -us
 \item -ismus
 \item -ekt
 \item -tor
 \item -stand
 \item -fall
 \end{itemize}
 Verbalstämme\\
 z.B. besuchen $\rightarrow$ der Besuch\\
 \\
 Verbalstämme + er\\
 z.B. lehren $\rightarrow$ der Lehrer\\

\end{minipage}
\begin{minipage}[t]{0,33\textwidth}
 \subsubsection{Weiblich}
 \begin{itemize}
 \item -ung
 \item -heit
 \item -keit
 \item -ion
 \item -schaft
 \item -ei
 \item -in
 \item -anz
 \item -enz
 \item -ät
 \item -e
 \item -ur
 \item -ie
 \item -ik
 \item -ek
 \item -iz
 \item -ion
 \item -sicht
 \item -kunft
 \item -schrift
 \end{itemize}
\end{minipage}
\begin{minipage}[t]{0,34\textwidth}
 \subsubsection{Sächlich}
 \begin{itemize}
 \item -um (maar niet -tum)
 \item -o
 \item -ment
 \item -nis
 \item GE -e
 \end{itemize}
 Verkleinwoorden:
 \begin{itemize}
 \item -chen
 \item -lein
 \end{itemize}
 Bruchzahlen:
 \begin{itemize}
 \item -tel
 \end{itemize}
\end{minipage}


\subsection{Betekenis}
Woorden die tot volgende groep behoren:\\
\\
\begin{minipage}[t]{0,33\textwidth}
 \subsubsection{Männlich}
 \begin{itemize}
 \item automerken
 \item maandnamen
 \item dagen van de week
 \item windrichtingen
 \item munteenheden
 \item oceanen	
 \item weersverschijnselen
 \end{itemize}
\end{minipage}
\begin{minipage}[t]{0,33\textwidth}
 \subsubsection{Weiblich}
 \begin{itemize}
 \item Europese rivieren
 \item vliegtuignamen
 \item scheepsnamen
 \item ruimtetuigen
 \item getallen
 \end{itemize}
\end{minipage}
\begin{minipage}[t]{0,34\textwidth}
 \subsubsection{Sächlich}
 \begin{itemize}
 \item letters alfabet
 \end{itemize}
\end{minipage}
\subsection{Vergelijk Nederlands-Duits}
In sommige gevalllen worden vorige regels overtroffen door het biologische gelsacht van het woord. Indien er geen biologisch gelsacht beschikbaar is en het ook niet duidelijk is met vorige regels dan kan men kijken naar het geslacht in het Nederlands. De meesten woorden komen overeen maar er zijn natuurlijk uitzonderingen:\\

\dictentry{het afscheid}{}{}{}{der Abschied}{}{}{}
\dictentry{het bedrag}{}{}{}{der Betrag}{}{}{}
\dictentry{het artikel}{}{}{}{der Artikel}{}{}{}
\dictentry{het bedrijf}{}{}{}{der Betrieb}{}{}{}
\dictentry{de telefoon}{}{}{}{das Telefon}{}{}{}
\dictentry{het apparaat}{}{}{}{der Apparat}{}{}{}
\dictentry{het getal}{}{}{}{die Zahl}{}{}{}
\dictentry{het bankroet}{}{}{}{der Bankrott}{}{}{}
\dictentry{het krediet}{}{}{}{der Kredit}{}{}{}
\dictentry{het loon}{}{}{}{der Lohn}{}{}{}
\dictentry{het ogenblik}{}{}{}{der Augenblick}{}{}{}
\dictentry{het contract}{}{}{}{der Kontrakt}{}{}{}
\dictentry{het concern}{}{}{}{der Konzern}{}{}{}
\dictentry{de maat}{}{}{}{das Maß}{}{}{}
\dictentry{het compromis}{}{}{}{der Kompromiss}{}{}{}
\dictentry{het transport}{}{}{}{der Transport}{}{}{}
\dictentry{het profijt}{}{}{}{der Profit}{}{}{}
\dictentry{het verschil}{}{}{}{der Unterschied}{}{}{}
\dictentry{het moment}{}{}{}{der Moment}{}{}{}
\dictentry{het verkeer}{}{}{}{der Verkehr}{}{}{}
\dictentry{het verlies}{}{}{}{der Verlust}{}{}{}
\dictentry{het bewijs}{}{}{}{der Beweis}{}{}{}
\dictentry{het gebruik}{}{}{}{der Gebrauch}{}{}{}
\dictentry{het antwoord}{}{}{}{die Antwort}{}{}{}
\dictentry{het plan}{}{}{}{der Plan}{}{}{}
\dictentry{het cijfer}{}{}{}{die Ziffer}{}{}{}
\dictentry{het geduld}{}{}{}{die Geduld}{}{}{}
\dictentry{het nummer}{}{}{}{die Nummer}{}{}{}
\dictentry{het succes}{}{}{}{der Erfolgt}{}{}{}
\dictentry{het werk}{}{}{}{die Arbeit}{}{}{}
\dictentry{het gevaar}{}{}{}{die Gefahr}{}{}{}
\dictentry{het verdrag}{}{}{}{der Vertrag}{}{}{}
\dictentry{het verdrag/contract}{}{}{}{der Kontrakt}{}{}{}
\dictentry{het park}{}{}{}{der Park}{}{}{}
\dictentry{het bezit}{}{}{}{der Besitz}{}{}{}
\dictentry{het bevel}{}{}{}{der Befehl}{}{}{}

\subsection{De vrouwelijke vorm van een mannelijk woord}
Om een mannelijk woord om te zetten naar een vrouwelijk wordt er "+IN aan toegevoegd.:\\
z.B.: \\
der Artz $\rightarrow$ die Ärztin\\
der Expert $\rightarrow$ die Expertin

\subsection{Betekenis bepaald door geslacht}
\dictentry{het loon}{}{}{}{das Gehalt}{}{}{}
\dictentry{gehalte}{}{}{}{der Gehalt}{}{}{}
\dictentry{het mes}{}{}{}{das Messer}{}{}{}
\dictentry{de meter}{}{}{}{der Messer}{}{}{}
\dictentry{het meer}{}{}{}{der See (n)}{}{}{}
\dictentry{de zee}{}{}{}{die See (n)}{}{}{}
\dictentry{de zee}{}{}{}{das Meer (e)}{}{}{}
\dictentry{het stuur}{}{}{}{das Steuer}{}{}{}
\dictentry{de belasting}{}{}{}{die Steuer}{}{}{}
\dictentry{deel on}{}{}{}{der Teil}{}{}{}
\dictentry{voordeel}{}{}{}{der Vorteil}{}{}{}
\dictentry{nadeel}{}{}{}{der Nachteil}{}{}{}
\dictentry{oordeel}{}{}{}{das Urteil}{}{}{}
\dictentry{het tegendeel}{}{}{}{das Gegenteil}{}{}{}
\dictentry{inkomen}{}{}{}{der Verdienst}{}{}{}
\dictentry{verdienste}{}{}{}{das Verdienst}{}{}{}
\dictentry{het lint}{}{}{}{das Band}{}{}{}
\dictentry{band, orkest}{}{}{}{die Band}{}{}{}
\dictentry{boekdeel}{}{}{}{der Band}{}{}{}
\dictentry{leider}{}{}{}{der Leiter}{}{}{}
\dictentry{ladder}{}{}{}{die Leiter}{}{}{}

\section{Die Deklination im Singular}
\subsection{Zwakke mannelijke zelfstandige naamwoorden}

\begin{tabular}{|c|c|}
\hline 
\rowcolor{gray}
 & männlich \\ 
\hline 
\cellcolor[gray]{0.8}Nominativ & $+\phi$ \\ 
\hline 
\cellcolor[gray]{0.8}Akkusativ & +(e)n \\ 
\hline 
\cellcolor[gray]{0.8}Dativ & +(e)n\\ 
\hline 
\cellcolor[gray]{0.8}Genitiv & +(e)n \\ 
\hline 
\end{tabular} 

\subsection{Sterke mannelijke zelfstandige naamwoorden}
\begin{tabular}{|c|c|c|}
\hline 
\rowcolor{gray}
& 1 lettergreep & $>1$ lettergrepen \\ 
\hline 
\cellcolor[gray]{0.8}Nominativ & $\phi$ & $\phi$\\ 
\hline 
\cellcolor[gray]{0.8}Akkusativ & $\phi$ & $\phi$\\ 
\hline 
\cellcolor[gray]{0.8}Dativ & (e) & $\phi$ \\ 
\hline 
\cellcolor[gray]{0.8}Genitiv & -es & -s \\ 
\hline 
\end{tabular} 

\subsection{Gemengde mannelijke zelfstandige naamwoorden}
\begin{minipage}[t]{0,33\textwidth}
\begin{tabular}{|c|c|}
\hline 
\rowcolor{gray}
 & männlich \\ 
\hline 
\cellcolor[gray]{0.8}Nominativ & $+\phi$ \\ 
\hline 
\cellcolor[gray]{0.8}Akkusativ & +n \\ 
\hline 
\cellcolor[gray]{0.8}Dativ & +n\\ 
\hline 
\cellcolor[gray]{0.8}Genitiv & +ns \\ 
\hline 
\end{tabular} 
\end{minipage}
\begin{minipage}{0,67\textwidth}
Op deze mannier zijn er maar 6 woorden die zich zo verbuigen:
\begin{enumerate}
\item der Name $\leftrightarrow$ de naam
\item der Friede $\leftrightarrow$ de vrede
\item der Buchstabe $\leftrightarrow$ de letter
\item der Glaube $\leftrightarrow$ het geloof
\item der Gedanke $\leftrightarrow$ de gedachte
\item der Wille $\leftrightarrow$ de wil
\end{enumerate}
\end{minipage}

\subsection{Vrouwelijke zelfstandige naamwoorden}
\begin{tabular}{|c|c|}
\hline 
\rowcolor{gray}
 & männlich \\ 
\hline 
\cellcolor[gray]{0.8}Nominativ & $+\phi$ \\ 
\hline 
\cellcolor[gray]{0.8}Akkusativ & $+\phi$ \\ 
\hline 
\cellcolor[gray]{0.8}Dativ & $+\phi$ \\ 
\hline 
\cellcolor[gray]{0.8}Genitiv & $+\phi$ \\ 
\hline 
\end{tabular}

\subsection{(Sterke) Onzijdige zelfstandige naamwoorden}
\begin{minipage}[t]{0.5\textwidth}
\begin{tabular}{|c|c|c|}
\hline 
\rowcolor{gray}
& 1 lettergreep & $>1$ lettergrepen \\ 
\hline 
\cellcolor[gray]{0.8}Nominativ & $\phi$ & $\phi$\\ 
\hline 
\cellcolor[gray]{0.8}Akkusativ & $\phi$ & $\phi$\\ 
\hline 
\cellcolor[gray]{0.8}Dativ & $\phi$(e) & $\phi$ \\ 
\hline 
\cellcolor[gray]{0.8}Genitiv & -es & -s \\ 
\hline 
\end{tabular} 
\end{minipage}
\begin{minipage}{0.5\textwidth}
Er is 1 uitzondering op deze verbruiging:\\
\begin{tabular}{|c|c|}
\hline 
\cellcolor[gray]{0.8}Nominativ & das Herz \\ 
\hline 
\cellcolor[gray]{0.8}Akkusativ & das Herz \\ 
\hline 
\cellcolor[gray]{0.8}Dativ & dem Herzen \\ 
\hline 
\cellcolor[gray]{0.8}Genitiv & des Herzens \\ 
\hline 
\end{tabular}
\end{minipage}
\subsection{Sterke mannelijke en onzijdige zelfstandige naamworden die afwijken}
\begin{itemize}
\item Woorden die eindigen op een klinker: +s
\item Klemtoon achteraan en meerdere lettergrepen:	+es
\item Woorden eindigend op “is” klank (x, ß, eis...):	+es
\item Woorden op -ismus:					$\phi$
\item Woorden op -nis:				+ses
\item Afkortingen:						Ø
\end{itemize}

\subsection{Eigennamen in genitief enkelvoud}
Moeten ook verbogen worden tenzij er al een lidwoord voor staat dat verbogen is.
\subsection{Verkleinwoorden}
(") + chen\\
(") + lein\\
\\
Meestal -chen, hoor een beetje hoe het klinkt.
\section{Die Deklination im Plural}
\end{document}
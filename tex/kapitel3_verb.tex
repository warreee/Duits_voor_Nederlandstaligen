\documentclass[main.tex]{subfiles}

\begin{document}

\chapter{Das Verb}
\section{Das Präsens}
\begin{tabular}{|c|c|c|c|c|c|c|}
\hline 
\rowcolor{gray}
\setlength\extrarowheight{5pt}
 & A: MACHEN & B: HEIßEN & C: ARBEITEN & D: HELFEN & E: SEHEN & F: FAHREN \\ 
\hline 
ich & mache & heiße & arbeite & helfe & sehe & fahre \\ 
\hline 
du & machst & heißt & arbeitest & hilfst & siehst & fährst \\ 
\hline 
er/sie/es & macht & heißt & arbeitet & hilft & sieht & fährt \\ 
\hline 
wir & machen & heißen & arbeiten & helfen & sehen & fahren \\ 
\hline 
ihr & macht & heißt & arbeitet & helft & seht & fahrt \\ 
\hline 
sie & machen & heißen & arbeiten & helfen & sehen & fahren \\ 
\hline 
Sie & machen & heißen & arbeiten & helfen & sehen & fahren \\ 
\hline 
\end{tabular} 

\subsection{A - De standaard vervoeging}

\subsection{B - Woorden eindigend op een sis-klank}
-x, -s, -ss, z niet met een extra +s

\subsection{C - Interpolatie}
Alle woorden eindigend op -d of -t.
\\
Alle woorden eindigend op -m of -n als ze niet zijn voorgegaan door -l of -r en bij dubbel m of dubbel n, als alle letters worden uitgesporken.

\begin{tabular}{|c|c|}
\hline 
atmen & inademen \\ 
\hline 
widmen & opdragen, wijden \\ 
\hline 
begegnen & tegenkomen \\ 
\hline 
ebnen & effenen \\ 
\hline 
(sich) eignen & geschikt zijn \\ 
\hline 
leugnen & ontkennen \\ 
\hline 
öffnen & openen \\ 
\hline 
ordnen & ordenen \\ 
\hline 
rechnen & rekenen \\ 
\hline 
regnen & regenen \\ 
\hline 
segnen & zegenen \\ 
\hline 
trocknen & drogen \\ 
\hline 
vervollkommnen & compleet maken \\ 
\hline 
waffnen & wapenen? \\ 
\hline 
(sich) wappnen & zich wapenen \\ 
\hline 
zeichnen & tekenen \\ 
\hline 
\end{tabular} 

\subsection{D - Korte  e }
\begin{minipage}{0.45\textwidth}
e wordt i in de 2e en 3e pers. enkelvoud.\\
Principe van B geldt hier ook.\\
Deze regel is sterker als C, dus indien er al klankverwisseling is gebeurd dan geen interpolatie meer.\\
Pas op voor, lange e maar toch met korte e vervoegd:\\
\begin{itemize}
\item treten: du trittst, er tritt, ihr tretet
\item gelten: du giltst, er gilt, ihr geltet
\item nehmen: du nimmst, er nimmt, ihr nehmt
\end{itemize}
\end{minipage}
\begin{minipage}{0.15\textwidth}
\end{minipage}
\begin{minipage}{0.40\textwidth}
\begin{tabular}[right]{|c|c|}
\hline 
bergen & bergen, bevatten \\ 
\hline 
brechen & breken \\ 
\hline 
erschrecken & schrikken \\ 
\hline 
essen & eten \\ 
\hline 
geben & geven \\ 
\hline 
gelten & geldig zijn \\ 
\hline 
helfen & helpen \\ 
\hline 
messen & meten \\ 
\hline 
nehmen & nemen \\ 
\hline 
sprechen & spreken \\ 
\hline 
stechen & steken \\ 
\hline 
sterben & sterven \\ 
\hline 
treffen & treffen \\ 
\hline 
treten & trappen \\ 
\hline 
verderben & verderven \\ 
\hline 
vergessen & vergeten \\ 
\hline 
werben & werven \\ 
\hline 
werfen & werpen \\ 
\hline 
\end{tabular} 
\end{minipage}

\subsection{E - Lange e}
\begin{minipage}{0.5\textwidth}
lange e wordt ie in de 2e en 3e pers. enkelvoud.\\
Principe van B geldt hier ook.\\
De volgende sterke werkwoorden hebben geen klankverwisseling:\\
\begin{tabular}{|c|c|}
\hline 
bewegen & bewegen \\ 
\hline 
gehen & gehen \\ 
\hline 
heben & opheffen \\ 
\hline 
stehen & staan \\ 
\hline 
genesen & genezen \\ 
\hline 
\end{tabular} 
\end{minipage}
\begin{minipage}{0.5\textwidth}
\begin{tabular}{|c|c|}
\hline 
befehlen & bevelen \\ 
\hline 
empfehlen & aanbevelen \\ 
\hline 
geschehen & gebeuren (enkel 3e pers) \\ 
\hline 
lesen & lezen \\ 
\hline 
sehen & zien \\ 
\hline 
stehlen & stelen \\ 
\hline 
\end{tabular} 
\end{minipage}
\subsection{F - Korte en lange a}
\begin{minipage}{0.5\textwidth}
a wordt ä.\\
Principe van B geldt hier ook.\\
De klankerverandering is sterker als het principe van C.\\
Speciaal geval stoßen: du stößt, er stößt\\
\\
Uitzondering: schaffen: du schaffst, er schafft		(lukken, brengen)
\end{minipage}
\begin{minipage}{0.5\textwidth}
\begin{tabular}{|c|c|}
\hline 
backen & bakken \\ 
\hline 
blasen & blazen \\ 
\hline 
empfangen & ontvangen \\ 
\hline 
fahren & besturen, rijden \\ 
\hline 
fallen & vallen \\ 
\hline 
fangen & vangen \\ 
\hline 
graben & graven \\ 
\hline 
halten & houden \\ 
\hline 
laden & laden \\ 
\hline 
lassen & laten \\ 
\hline 
laufen & lopen \\ 
\hline 
raten & raden \\ 
\hline 
schlafen & slapen \\ 
\hline 
schlagen & slagen, vechten \\ 
\hline 
tragen & dragen \\ 
\hline 
wachsen & groeien \\ 
\hline 
waschen & wassen \\ 
\hline 
\end{tabular} 
\end{minipage}

\section{Das Imperfekt}
\begin{tabular}{|c|c|c|c|c|}
\hline 
\rowcolor{gray}
\setlength\extrarowheight{5pt}
 & A1: MACHEN & A2: ARBEITEN & B: HELFEN & C: BRINGEN \\ 
\hline 
ich & machte & arbeitete & half & brachte \\ 
\hline 
du & machtest & arbeitest & half & brachtest \\ 
\hline 
er/sie/es & machte & arbeitete & half & brachte \\ 
\hline 
wir & machten & arbeiteten & halfen & brachten \\ 
\hline 
ihr & machtet & arbeitetet & halft & brachtet \\ 
\hline 
sie & machten & arbeiteten & halfen & brachten \\ 
\hline 
Sie & machten & arbeiteten & halfen & brachten \\ 
\hline 
\end{tabular} 
\subsection{A: Schwache Verben}
\subsubsection{Uitgangen}
\begin{tabular}{c c c}

ich & $\leftrightarrow$ & te \\ 

du & $\leftrightarrow$ & test \\ 

er/sie/es & $\leftrightarrow$ & te \\ 

wir & $\leftrightarrow$ & ten \\ 

ihr & $\leftrightarrow$ & tet \\ 

sie & $\leftrightarrow$ & ten \\ 
 
Sie & $\leftrightarrow$ & ten \\ 

\end{tabular} \\
\\
Het principe van sis-klank blijft gelden.
\subsubsection{A2}
Dit zijn de werkwoorden waarvan de stam eindigt op:
\begin{itemize}
\item d/t
\item m/n (maar niet voorafgegaan door l/r)
\item m/n (maar niet voorafgegaan door m/n indien beide worden uitgesproken)
\end{itemize}
\todo[inline]{nagaan dubbele m/n}
Hier treedt interpolatie op in elke vorm!

\subsubsection{Sterk in het Nederlands maar zwak in het Duits:}
\begin{tabular}{ccc}
vroeg & $\leftrightarrow$ & fragte \\ 
kocht & $\leftrightarrow$ & dachte \\ 
kreeg & $\leftrightarrow$ & kriegte \\ 
schonk & $\leftrightarrow$ & schenkte \\ 
dook & $\leftrightarrow$ & tauchte \\ 
zocht & $\leftrightarrow$ & suchte \\ 
\end{tabular} 
\subsection{B: Starke Verben}
De meeste Duitse sterke werkwoorden komen overeen met de Nederlandse. Zie lijst voor meer details.
\subsubsection{Uitgangen}
\begin{tabular}{c c c}

ich & $\leftrightarrow$ & $\phi$ \\ 

du & $\leftrightarrow$ & st \\ 

er/sie/es & $\leftrightarrow$ & $\phi$ \\ 

wir & $\leftrightarrow$ & en \\ 

ihr & $\leftrightarrow$ & t \\ 

sie & $\leftrightarrow$ & en \\ 
 
Sie & $\leftrightarrow$ & en \\ 

\end{tabular}
\\
\\
Het principe van sis-klank blijft gelden.\\
De tweede persoon meervoud +e indien de Imperfektstam op d of t eindigt.
\subsection{Unregelmäßige Verben}
Hierbij zijn 2 grote categorieën te onderscheiden:
\subsubsection{Onveranderde stam met sterke uitgangen}
z.B. gehen:
\begin{tabular}{cc}

ich & ging \\ 

du & gingst \\ 

er/sie/es & ging \\ 

wir & gingen \\ 

ihr & gingt \\ 
 
sie & gingen \\ 
Sie & gingen \\ 
\end{tabular} 
\subsubsection{Veranderde stam met zwakke uitgangen}
Hetzelfde principe als in het voorbeeld hierboven maar dan omgekeerd.\\
Veel gebruikte woorden van dit type:\\
\begin{tabular}{c c c}

bringen & $\leftrightarrow$ & brachte \\ 

denken & $\leftrightarrow$ & dachte \\ 

brennen & $\leftrightarrow$ & brannte \\ 

kennen & $\leftrightarrow$ & kannte \\ 

nennen & $\leftrightarrow$ & nannte \\ 

rennen & $\leftrightarrow$ & rannte \\ 
 
senden & $\leftrightarrow$ & sandte (sendete) \\ 

wenden & $\leftrightarrow$ & wandte (wendete) \\ 
\end{tabular}
\section{Das Perfekt}
\subsection{Vorming}
Ook hier kunnen we de werkwoorden weer onderbrengen in 3 categorieën: sterke, zwakke, gemengde. Voor de vorming maakt dit niet zo veel verschil. Bij de zwakke komt de stam uit de o.t.t. en voor de andere komt die uit o.v.t.:
\\
\\
\textbf{GE* + STAM + T/ET/EN}
\\
\\ * GE wordt enkel toegvoegd indien de klemtoon niet op de laatste lettergreep ligt.
\subsection{Het hulpwerkwoord}
De hulpwerkwoorden zijn net als in het Nederlands \textit{haben} en \textit{sein}.
Ze komen in de meeste gevallen overeen maar er zijn enkele uitzonderingen:
\begin{itemize}
\item begonnen / angefangen
\item zugenommen / abgenommen
\item vergessen
\item verheiratet
\item aufgehört
\item geendet
\item gefallen
\end{itemize}
Maar ook bewegingswerkwoorden:
\begin{itemize}
\item begegnet (tegenkomen)
\item gefolgt
\item gelfogen
\item geschwommen
\item gelaufen
\end{itemize}
\subsubsection{Modale verben}
Gebruiken altijd haben.
\section{Das Futur}
Het gaat hier om de neutrale toekomst, gehen of sollen wordt hiervoor niet gebruikt. Het is de equivalen van zullen in het Nederlands.
\subsection{Vorming}
\textbf{Futur 1 = Präsens von «werden» + Infinitiv}
\section{Das Passiv}
\subsection{Van actief naar passief}
\begin{tabular}{c c c}

\textbf{AKTIV} & $\rightarrow$ & \textbf{PASSIV} \\ 
\hline
\textbf{Subjekt} & $\rightarrow$ & \textbf{von + Dativ (handeland voorwerp)}\\ 
\textbf{z.B.} Die Regierung reagierte nicht. & $\rightarrow$ & Von der Regierung wurde nicht reagiert. \\ 
\hline
\textbf{man (onbep. subj.)} & $\rightarrow$ & \textbf{Valt weg of: es} \\ 
\textbf{z.B.} Hier sagte man nichts & $\rightarrow$ & Hier wurde nichts gesagt. \\ 
\textbf{z.B.} Hier sagte man nichts & $\rightarrow$ & Es wurde hier nichts gesagt. \\ 
\hline
\textbf{lijdend voorwerp} & $\rightarrow$ & \textbf{Nominativ} \\ 

\textbf{z.B.} Man schlug ihn ins Gesicht. & $\rightarrow$ & Er wurde ins Gesicht geschlagen. \\ 
\hline
\textbf{Dativ} & $\rightarrow$ & \textbf{Dativ} \\ 

\textbf{z.B.} So schadet man keinem. & $\rightarrow$ & So wird keinem geschadet. \\ 

\end{tabular}
\\
Het handelend voorwerp kan ook net zoals in het Nederlans met \textbf{durch + ACC} gevorm worden. Er is dan echter wel een betekenisverschil. Er moet in tegenstelling tot \textbf{von + DAT} een 3e partij betrokken zijn.

\todo[inline]{Nagaan wat er bedoelt wordt met puntje 6 in boek}
\subsection{Vorming}
Die passende Form von «werden» + Partizip Perfekt des Hauptverbs.
\subsubsection{Worden}
Worden wordt soms toegevoegd in de passief met sein na het hoofdwerkwoord:\\
\\
NIET: de toestand wordt beschreven: die Tür ist geschlossen.\\
WEL: de handeling wordt beschreven: die Tür ist (gestern/von mir) geschlossen worden.
\subsubsection{Geworden}
Naast worden bestaan er geworden in het Duits. Dit wordt wel enkel in het actief gebruikt en wordt steeds voorafgegaan door een van de volgende:
\begin{itemize}
\item Adjectiv
\item telwoord
\item Substantiv
\end{itemize}
\subsubsection{Modalverben}
Deze hebben geen passieve vorm, ze staan in de Infitiv, daarna volgt het voltooid deelwoord en als laatste de infitief werden.\\
\textbf{z.B.}\\
Man kann sie besiegen. $\rightarrow$ Sie können besiegt werden.
\section{Die modale Verben}
\subsection{Vorming}

\subsection{Vertaling}

\begin{enumerate}[label=\Alph*]
\item wissen = weten
\item können = kunnen
\item wollen = willen
\item müssen = moeten
	\begin{enumerate}
	\item Einladung/Empfehlung (persoon wordt direct aangesproken)
	\item logische Schlussfolgerung
	\item = 4 objektive Verpflichtung (op basis van omstandigheden, niet opgelegd door iemand anders, door jezelf mag wel)	
	\end{enumerate}
\item sollen = moeten
	\begin{enumerate}
	\item subjektive Verpflichtung=Gebot
	\item potentielle subjektive Verpflichtung (zou je graag hebben dat)
	\item sollte=Konjunktiv II = zou moeten/zou beter
 	 = indirekte subjektive Verpflichtung / Vorwurf(verwijt)
 	 \item moralische subjektive V.
 	 \item kollektive subjektive V.
 	 \item Zweifel/Unsicherheit (vaak met wissen)
 	 \item Plan (enkel zaken kunnen gepland worden, maar wel met baby's, de baby zal volgende week geboren worden, het is gepland)
	 \item Gerücht $\rightarrow$ zou, komt vaak voor in de tegenwoordige tijd
	 \item sollte = Konjucktiv II $\rightarrow$ zou (onvoltooid verleden toekomende tijd)
	 \item sollte = Konjunktiv II $\rightarrow$ Hypothese
	$\rightarrow$ mocht/moest 	
	$\rightarrow$ zou 
	\end{enumerate}
\item dürfen = mogen $\rightarrow$ Zulassung
	\begin{enumerate}
	\item ich darf=soll nicht ausgehen
	\end{enumerate}
\item mögen = graag hebben
	\begin{enumerate}
	\item nicht aktueller Wunsch = gern haben/trinken/essen
	\item aktueller Wunsch $\rightarrow$ möchte : ik zou graag … = Konjunktiv II
	\end{enumerate}
\end{enumerate}

\section{Der Imperativ}
\subsection{Singular}
\begin{enumerate}
\item \textbf{Algemein:} Stamm (+e indien er interpolatie zou zijn)

\item \textbf{e/i-Wechsel:} a/ä nicht
	\begin{itemize}
	\item sprechen $\rightarrow$ sprich!
	\item fahren $\rightarrow$ fahr!
	\item stehlen $\rightarrow$ Stiehl!
	\end{itemize}

\item \textbf{Verben auf -eln}
	\begin{itemize}
	\item klingeln $\rightarrow$ klingle!
	\end{itemize}
\end{enumerate}
\subsubsection{Ausnahme}
Sein $\rightarrow$ sei!
\subsection{Plural}
Stamm + (e indien er interpolatie zou zijn) t
\subsubsection{Ausnahme}
Sein $\rightarrow$ seid!
\subsection{Höflichkeitsform}
\textbf{Infinitiv + Sie}

\section{Der Konjunktiv}
\subsection{Gebruik}
\todo[inline]{Vragen voor het gebruik van deze vorm, en wanneer er altijd met wurden mag gewerkt worden}
\subsection{Hoofdwerkwoorden}
\subsubsection{Zwakke werkwoorden}
\begin{tabular}{cc}

ich & würde \\ 

du & würdest \\ 

er/sie/es & würde \\ 

wir & würden \\ 

ihr & würdet \\ 
 
sie & würden \\ 
Sie & würden  
\end{tabular} 
+INFINITIV
\subsubsection{Sterke werkwoorden}
IMPERFEKSTAMM + " +
\begin{tabular}{c} 
					e\\
					est\\
					e\\
					en\\
					et\\
					en\\
					en\\
									
					
\end{tabular}\\
Indien er geen " \ op de stam kan worden gezet bvb. bij steigen (stieg) dan moet in de 1e en 3e persoon meervoud WÜRDEN + INFINITIV gebruikt worden (om verwarring met de ovt te vermijden).
\subsubsection{Onregelmatige werkwoorden}
Zelfde vorming als de zwakke werkwoorden.
\subsection{Hulpwerkwoorden + modale werkwoorden (+wissen)}
\subsubsection{haben/sein/werden}
Zie appendix
\subsubsection{Modale werkwoorden + wissen}
OVT + Umlaut\\
\\
Behalve voor sollen en wollen, hierbij is ovt = konjunktiv!

\end{document}
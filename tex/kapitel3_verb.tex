\documentclass[main.tex]{subfiles}

\begin{document}

\chapter{Das Verb}
\section{Das Präsens}
\begin{tabular}{|c|c|c|c|c|c|c|}
\hline 
\rowcolor{gray}
\setlength\extrarowheight{5pt}
 & A: MACHEN & B: HEIßEN & C: ARBEITEN & D: HELFEN & E: SEHEN & F: FAHREN \\ 
\hline 
ich & mache & heiße & arbeite & helfe & sehe & fahre \\ 
\hline 
du & machst & heißt & arbeitest & hilfst & siehst & fährst \\ 
\hline 
er/sie/es & macht & heißt & arbeitet & hilft & sieht & fährt \\ 
\hline 
wir & machen & heißen & arbeiten & helfen & sehen & fahren \\ 
\hline 
ihr & macht & heißt & arbeitet & helft & seht & fahrt \\ 
\hline 
sie & machen & heißen & arbeiten & helfen & sehen & fahren \\ 
\hline 
Sie & machen & heißen & arbeiten & helfen & sehen & fahren \\ 
\hline 
\end{tabular} 

\subsection{A - De standaard vervoeging}

\subsection{B - Woorden eindigend op een sis-klank}
-x, -s, -ss, z niet met een extra +s

\subsection{C - Interpolatie}
Alle woorden eindigend op -d of -t.
\\
Alle woorden eindigend op -m of -n als ze niet zijn voorgegaan door -l of -r en bij dubbel m of dubbel n, als alle letters worden uitgesporken.

\begin{tabular}{|c|c|}
\hline 
atmen & inademen \\ 
\hline 
widmen & opdragen, wijden \\ 
\hline 
begegnen & tegenkomen \\ 
\hline 
ebnen & effenen \\ 
\hline 
(sich) eignen & geschikt zijn \\ 
\hline 
leugnen & ontkennen \\ 
\hline 
öffnen & openen \\ 
\hline 
ordnen & ordenen \\ 
\hline 
rechnen & rekenen \\ 
\hline 
regnen & regenen \\ 
\hline 
segnen & zegenen \\ 
\hline 
trocknen & drogen \\ 
\hline 
vervollkommnen & compleet maken \\ 
\hline 
waffnen & wapenen? \\ 
\hline 
(sich) wappnen & zich wapenen \\ 
\hline 
zeichnen & tekenen \\ 
\hline 
\end{tabular} 

\subsection{D - Korte  e }
\begin{minipage}{0.45\textwidth}
e wordt i in de 2e en 3e pers. enkelvoud.\\
Principe van B geldt hier ook.\\
Deze regel is sterker als C, dus indien er al klankverwisseling is gebeurd dan geen interpolatie meer.\\
Pas op voor, lange e maar toch met korte e vervoegd:\\
\begin{itemize}
\item treten: du trittst, er tritt, ihr tretet
\item gelten: du giltst, er gilt, ihr geltet
\item nehmen: du nimmst, er nimmt, ihr nehmt
\end{itemize}
\end{minipage}
\begin{minipage}{0.15\textwidth}
\end{minipage}
\begin{minipage}{0.40\textwidth}
\begin{tabular}[right]{|c|c|}
\hline 
bergen & bergen, bevatten \\ 
\hline 
brechen & breken \\ 
\hline 
erschrecken & schrikken \\ 
\hline 
essen & eten \\ 
\hline 
geben & geven \\ 
\hline 
gelten & geldig zijn \\ 
\hline 
helfen & helpen \\ 
\hline 
messen & meten \\ 
\hline 
nehmen & nemen \\ 
\hline 
sprechen & spreken \\ 
\hline 
stechen & steken \\ 
\hline 
sterben & sterven \\ 
\hline 
treffen & treffen \\ 
\hline 
treten & trappen \\ 
\hline 
verderben & verderven \\ 
\hline 
vergessen & vergeten \\ 
\hline 
werben & werven \\ 
\hline 
werfen & werpen \\ 
\hline 
\end{tabular} 
\end{minipage}

\subsection{E - Lange e}
\begin{minipage}{0.5\textwidth}
lange e wordt ie in de 2e en 3e pers. enkelvoud.\\
Principe van B geldt hier ook.\\
De volgende sterke werkwoorden hebben geen klankverwisseling:\\
\begin{tabular}{|c|c|}
\hline 
bewegen & bewegen \\ 
\hline 
gehen & gehen \\ 
\hline 
heben & opheffen \\ 
\hline 
stehen & staan \\ 
\hline 
genesen & genezen \\ 
\hline 
\end{tabular} 
\end{minipage}
\begin{minipage}{0.5\textwidth}
\begin{tabular}{|c|c|}
\hline 
befehlen & bevelen \\ 
\hline 
empfehlen & aanbevelen \\ 
\hline 
geschehen & gebeuren (enkel 3e pers) \\ 
\hline 
lesen & lezen \\ 
\hline 
sehen & zien \\ 
\hline 
stehlen & stelen \\ 
\hline 
\end{tabular} 
\end{minipage}
\subsection{F - Korte en lange a}
\begin{minipage}{0.5\textwidth}
a wordt ä.\\
Principe van B geldt hier ook.\\
De klankerverandering is sterker als het principe van C.\\
Speciaal geval stoßen: du stößt, er stößt\\
\\
Uitzondering: schaffen: du schaffst, er schafft		(lukken, brengen)
\end{minipage}
\begin{minipage}{0.5\textwidth}
\begin{tabular}{|c|c|}
\hline 
backen & bakken \\ 
\hline 
blasen & blazen \\ 
\hline 
empfangen & ontvangen \\ 
\hline 
fahren & besturen, rijden \\ 
\hline 
fallen & vallen \\ 
\hline 
fangen & vangen \\ 
\hline 
graben & graven \\ 
\hline 
halten & houden \\ 
\hline 
laden & laden \\ 
\hline 
lassen & laten \\ 
\hline 
laufen & lopen \\ 
\hline 
raten & raden \\ 
\hline 
schlafen & slapen \\ 
\hline 
schlagen & slagen, vechten \\ 
\hline 
tragen & dragen \\ 
\hline 
wachsen & groeien \\ 
\hline 
waschen & wassen \\ 
\hline 
\end{tabular} 
\end{minipage}
\section{Der Imperativ}
\subsection{Singular}
\begin{enumerate}
\item \textbf{Algemein:} Stamm (+e)

\item \textbf{e/i-Wechsel:} a/ä nicht
	\begin{itemize}
	\item sprechen $\rightarrow$ sprich!
	\item fahren $\rightarrow$ fahr!
	\item stehlen $\rightarrow$ Stiehl!
	\end{itemize}

\item \textbf{Verben auf -eln}
	\begin{itemize}
	\item klingeln $\rightarrow$ klingle!
	\end{itemize}
\end{enumerate}
\subsubsection{Ausnahme}
Sein $\rightarrow$ sei!
\todo[inline]{Vragen: wanner +e}
\subsection{Plural}
\textbf{Stamm + (e*)t}
\subsubsection{Ausnahme}
Sein $\rightarrow$ seid!
\subsection{Höflichkeitsform}
\textbf{Infinitiv + Sie}
\end{document}
\documentclass[main.tex]{subfiles}

\begin{document}

\chapter{Das Pronomen}
%%%%%%%%%%%%%%%%%%%%%%%%%%%%%%%%%%%%%%%%%%%%%%%%%%%%%%%%%%%%%%%%%%%%%%%%%%%%%%%%%%%%%
%%%%%%%%%%%%%%%%%%%%%%%%%%%%%%%%%%%%%%%%%%%%%%%%%%%%%%%%%%%%%%%%%%%%%%%%%%%%%%%%%%%%%
\section{Personalpronomen}
\begin{tabular}{|c|c|c|}
\rowcolor{gray}
\setlength\extrarowheight{8pt}
Nominativ & Akkusativ & Dativ \\
\hline 
ich & mich & mir \\ 
du & dich & dir \\ 
Sie & Sie & Ihnen \\ 
er & ihn & ihm \\ 
sie & sie & ihr \\ 
es & es & ihm \\ 
man / einer / frau & einen & einem\\
wir & uns & uns \\ 
ihr & euch & euch \\ 
sie & sie & ihnen \\ 
\hline
\end{tabular} \\
\\
In het Duits is je en men niet verwisselbaar.\\
\textit{Es} heeft 2 betekenissen:
\begin{enumerate}
\item \textit{het:} \todo[inline]{Waneer wordt deze vertaling gebruikt?}
\item \textit{er:} es wordt enkel in het Duits geschreven als het vlak voor het vervoegde werkwoord staat. Behalve bij gibt, hier wordt het altijd geschreven.
\end{enumerate}

%%%%%%%%%%%%%%%%%%%%%%%%%%%%%%%%%%%%%%%%%%%%%%%%%%%%%%%%%%%%%%%%%%%%%%%%%%%%%%%%%%%%%
%%%%%%%%%%%%%%%%%%%%%%%%%%%%%%%%%%%%%%%%%%%%%%%%%%%%%%%%%%%%%%%%%%%%%%%%%%%%%%%%%%%%%
\section{Reflexivpronomen}
Wordt enkel gebruikt in accusatief en datief als Lv of Mv.\\
\begin{tabular}{|c|c|c|}
\rowcolor{gray}
\setlength\extrarowheight{8pt}
& Akkusativ & Dativ \\
\hline 
(ich) & mich & mir \\ 
(du) & dich & dir \\ 
(Sie) & sich & sich \\ 
(er) & sich & sich \\ 
(sie) & sich & sich \\ 
(es) & sich & sich \\ 
(wir) & uns & uns \\ 
(ihr) & euch & euch \\ 
(sie) & sie & ihnen \\ 
\hline
\end{tabular} 

%%%%%%%%%%%%%%%%%%%%%%%%%%%%%%%%%%%%%%%%%%%%%%%%%%%%%%%%%%%%%%%%%%%%%%%%%%%%%%%%%%%%%
%%%%%%%%%%%%%%%%%%%%%%%%%%%%%%%%%%%%%%%%%%%%%%%%%%%%%%%%%%%%%%%%%%%%%%%%%%%%%%%%%%%%%
\section{Possessivpronomen}
\begin{tabular}{|c|c|c|}
\rowcolor{gray}
(ich) & mein \\ 
(du) & dein \\ 
(Sie) & Ihr \\ 
(er) & sein \\ 
(sie) & ihr \\ 
(es) & sein \\ 
(wir) & unser \\ 
(ihr) & euer (*) \\ 
(sie) & ihr \\ 
\hline
\end{tabular} \\
\\
Deze vormen worden daarna vervoegd met de uitgangen van het onbepaald lidwoord ein.
Het geslacht van de stam wordt steeds bepaald door de bezitter, de uitgangen door het geslacht van het bezit.
\\
\\
(*) eurer + en $\rightarrow$ euren
%%%%%%%%%%%%%%%%%%%%%%%%%%%%%%%%%%%%%%%%%%%%%%%%%%%%%%%%%%%%%%%%%%%%%%%%%%%%%%%%%%%%%
%%%%%%%%%%%%%%%%%%%%%%%%%%%%%%%%%%%%%%%%%%%%%%%%%%%%%%%%%%%%%%%%%%%%%%%%%%%%%%%%%%%%%
\section{Fragepronomen}
\begin{tabular}{ccc}
wie viel & $\rightarrow$ & hoeveel \\ 
wann & $\rightarrow$ & wanneer \\ 
wohin & $\rightarrow$ & naar waar (van de spreker weg) \\ 
wofür & $\rightarrow$ & waarin \\ 
woher & $\rightarrow$ & van waar(naar de spreker toe) \\ 
warum & $\rightarrow$ & waarom \\ 
womit & $\rightarrow$ & waarmee \\ 
was & $\rightarrow$ & wat \\ 
wo & $\rightarrow$ & waar \\ 
\end{tabular} \\
\\
De vorige Fragepronomen worden niet verbogen, het Fragepronomen wer (wie) wordt wel verbogen:\\
\begin{tabular}{ccccc}
N & $\rightarrow$ & wer & $\rightarrow$ & wie\\
A & $\rightarrow$ & wen & $\rightarrow$ & wie\\
D & $\rightarrow$ & wem & $\rightarrow$ & wie\\
G & $\rightarrow$ & wessem & $\rightarrow$ & wiens\\
\end{tabular} 

\subsection{Voornaamwoordelijke bijwoorden}
wo of da: indien klinkers botsen +r
\begin{itemize}
\item wofür/womit/worin
\item dafür/damit/darin/daruber
\item hierfür/hiermit/hierin
\end{itemize}

\textit{worum} wordt enkel gebruikt met werkwoorden die um gebruiken

\textbf{wann 1} = Fragepronomen, fragt nach einem Zeitpunkt\\
\textbf{wann 2} = wanneer ook\\
Sie können mich anrufen, wann Sie wollen\\
\\
\textbf{wenn 1} = hypothetisch\\
$\rightarrow$ wenn es regnet bleiben wir zu Hause \\
\textbf{wenn 2} = repetitiv\\
wenn er seine Frau anrief, nahm sie nicht ab\\
\\
\textbf{als =} einmalig in de Vergangenheit -> toen!\\
als er seine Frau anrief, nahm sie nicht ab

%%%%%%%%%%%%%%%%%%%%%%%%%%%%%%%%%%%%%%%%%%%%%%%%%%%%%%%%%%%%%%%%%%%%%%%%%%%%%%%%%%%%%
%%%%%%%%%%%%%%%%%%%%%%%%%%%%%%%%%%%%%%%%%%%%%%%%%%%%%%%%%%%%%%%%%%%%%%%%%%%%%%%%%%%%%
\section{Relativpronomen}
\begin{tabular}{|c|c|c|c|c|}
\hline 
\rowcolor{gray}
& männlich & weiblich & sächlich & plural \\ 
\hline 
\cellcolor[gray]{0.8}Nominativ & der & die & das & die \\ 
\hline 
\cellcolor[gray]{0.8}Akkusativ & den & die & das & die \\ 
\hline 
\cellcolor[gray]{0.8}Dativ & dem & der & dem & denen \\ 
\hline 
\cellcolor[gray]{0.8}Genitiv & dessen & deren & dessen & deren \\ 
\hline 
\end{tabular} \\
Voor vorm stel volgende vragen:
\begin{enumerate}
\item Geslacht van de antecedent
\item Getal van de antecedent
\item Kasus is de functie in de bijzin
\end{enumerate}

\textbf{Das I:} lidwoord\\
\textbf{Das II:} Relativpronomen\\
\textbf{Dass:} voegwoord\\
Ezelsbruggetje: indien we met es kunnen vervangen gebruik das, anders dass.\\

\begin{tabular}{ccc}
womit & $\rightarrow$ & waarmee \\ 
was & $\rightarrow$ & wat (altijd hiermee vertaald) \\ 
woran & $\rightarrow$ & waaraan \\ 
\end{tabular} \\
\\
Na de volgende constructies komt was:
\begin{itemize}
\item vieles
\item das
\item na een gestubstantivieerd onzijdig adjectief: z.B. das beste was...
\end{itemize}
\end{document}
\documentclass[main.tex]{subfiles}

\begin{document}

\chapter{Das Adjektiv}
Er zijn twee types van verbuiging voor het adjectief: de sterke en zwakke verbuiging.\\
Ze komen voor in volgende drie gevallen:
\\
\\
\bgroup
\def\arraystretch{1.5}
\begin{tabular}{cccccccc}
1. & bepalend woord met naamvalsuitgang & + & adjectief & + & substantief & $\rightarrow$ & schwach \\ 

2. & bepalend woord zonder naamvalsuitgang & + & adjectief & + & substantief & $\rightarrow$ & stark \\

3. & geen bepalend woord & + & adjectief & + & substantief & $\rightarrow$ & stark \\ 
\end{tabular} 
\egroup

\section{Schwache Deklination}
\begin{tabular}{|c|c|c|c|c|c|c|c|c|}
\hline 
\rowcolor{gray}
  & männlich &  & weiblich &  & sächlich &  & Plural &  \\ 
\hline 
\cellcolor[gray]{0.8}Nominativ & der & -E & die/eine & -E & das & -E & die/keine & -EN \\ 
\hline 
\cellcolor[gray]{0.8}Akkusativ & den/einen & -EN & die/eine & -E & das & -E & die/keine & -EN \\ 
\hline 
\cellcolor[gray]{0.8}Dativ & dem/einem & -EN & der/einer & -EN & dem/einem & -EN & den/keinen & -EN \\ 
\hline 
\cellcolor[gray]{0.8}Genitiv & des/eines & -EN & der/einer & -EN & des/eines & -EN & der/keiner & -EN \\ 
\hline 
\end{tabular} 
\section{Starke Deklination}
\begin{tabular}{|c|c|c|c|c|c|c|c|c|}
\hline 
\rowcolor{gray}
  & männlich &  & weiblich &  & sächlich &  & Plural &  \\ 
\hline 
\cellcolor[gray]{0.8}Nominativ & $\phi$/ein & -E & $\phi$ & -E & $\phi$/ein & -E & $\phi$ & -E \\ 
\hline 
\cellcolor[gray]{0.8}Akkusativ & $\phi$ & -EN & $\phi$ & -E & $\phi$/ein & -E & $\phi$ & -E \\ 
\hline 
\cellcolor[gray]{0.8}Dativ & $\phi$ & -EN & $\phi$ & -EN & $\phi$ & -EN & $\phi$ & -EN \\ 
\hline 
\cellcolor[gray]{0.8}Genitiv & $\phi$ & \textcolor{red}{-EN} +ES & $\phi$ & -EN & $\phi$ & \textcolor{red}{-EN} +ES & $\phi$ & -ER \\ 
\hline 
\end{tabular} 

\section{Das substantivierte adjektiv}
\todo[inline]{Vragen: hoe worden ze nu effectief vervoegd, als (sterk) adjektief in elke naamval?}
\end{document}










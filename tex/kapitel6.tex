\documentclass[main.tex]{subfiles}

\begin{document}

\chapter{Das Adjektiv}
Er zijn twee types van verbuiging voor het adjectief: de sterke en zwakke verbuiging.\\
Ze komen voor in volgende drie gevallen:
\\
\\
\bgroup
\def\arraystretch{1.5}
\begin{tabular}{cccccccc}
1. & bepalend woord met naamvalsuitgang & + & adjectief & + & substantief & $\rightarrow$ & schwach \\ 

2. & bepalend woord zonder naamvalsuitgang & + & adjectief & + & substantief & $\rightarrow$ & stark \\

3. & geen bepalend woord & + & adjectief & + & substantief & $\rightarrow$ & stark \\ 
\end{tabular} 
\egroup

\section{Schwache Deklination}
\begin{tabular}{|c|c|c|c|c|c|c|c|c|}
\hline 
\rowcolor{gray}
  & männlich &  & weiblich &  & sächlich &  & Plural &  \\ 
\hline 
\cellcolor[gray]{0.8}Nominativ & der & -E & die/eine & -E & das & -E & die/keine & -EN \\ 
\hline 
\cellcolor[gray]{0.8}Akkusativ & den/einen & -EN & die/eine & -E & das & -E & die/keine & -EN \\ 
\hline 
\cellcolor[gray]{0.8}Dativ & dem/einem & -EN & der/einer & -EN & dem/einem & -EN & den/keinen & -EN \\ 
\hline 
\cellcolor[gray]{0.8}Genitiv & des/eines & -EN & der/einer & -EN & des/eines & -EN & der/keiner & -EN \\ 
\hline 
\end{tabular} 
\section{Starke Deklination}
\begin{tabular}{|c|c|c|c|c|c|c|c|c|}
\hline 
\rowcolor{gray}
  & männlich &  & weiblich &  & sächlich &  & Plural &  \\ 
\hline 
\cellcolor[gray]{0.8}Nominativ & $\phi$/ein & -ER & $\phi$ & -E & $\phi$/ein & -ES & $\phi$ & -E \\ 
\hline 
\cellcolor[gray]{0.8}Akkusativ & $\phi$ & -EN & $\phi$ & -E & $\phi$/ein & -ES & $\phi$ & -E \\ 
\hline 
\cellcolor[gray]{0.8}Dativ & $\phi$ & -EM & $\phi$ & -ER & $\phi$ & -EM & $\phi$ & -EN \\ 
\hline 
\cellcolor[gray]{0.8}Genitiv & $\phi$ & \textcolor{red}{-EN} +ES & $\phi$ & -ER & $\phi$ & \textcolor{red}{-EN} +ES & $\phi$ & -ER \\ 
\hline 
\end{tabular} 

\section{Das substantivierte adjektiv}

Woorden als:
\begin{itemize}
\item Schwarze(r)
\item Weise(r)
\item Verstorbene(r)
\end{itemize}

worden vervoegd volgens de regels van het adjectief, dus zowel sterk als zwak.

\section{Comparatief \& Superlatief}

\subsection{Standaard verbuiging}

$\phi \rightarrow$ +ER (als) $\rightarrow$ +ST

\subsubsection{a/o/u}
Deze hebben een umlaut in de vergrotende en overtreffende trap. z.B.:\\
arm $\rightarrow$ ärmer als $\rightarrow$ ärmst

\subsubsection{Achteraan beklemtoonde op t/d/s/ss/ß/x/sch}

Interpolatie treedt op z.B.:\\
leicht $\rightarrow$ leichter als $\rightarrow$ leichtest

\subsection{Onregelmatige verbuiging}

\begin{tabular}{ccccc}
gut & $\rightarrow$ & besser als & $\rightarrow$ & best \\ 
gern & $\rightarrow$ & lieber als & $\rightarrow$ & liebst \\ 
viel & $\rightarrow$ & mehr & als $\rightarrow$ & meist \\ 
\end{tabular} 

\subsection{Vergelijkingen}
Indien er moet vergeleken worden tussen verschillende zaken dan gebruiken we:\\
2 zaken $\rightarrow$ comparatief\\
$\geq$ 3 zaken $\rightarrow$ superlatief

\subsection{Nominatieve superlatief}
Am + Superlativ + en\\
z.B.: am schnellsten

\section{Onvervoegde adjectieven}
\begin{itemize}
\item steden: New Yorker \\
De inwoners: +ER(IN)
\todo[inline]{worden inwoners wel vervoegd?}
\item getallen: zwanziger
\end{itemize}

\section{Hoofdletter}
Indien het adjectief bij een eigennaam hoort dan met hoofdletter, anders zonder.\\
z.B. das Rote Kreutz $\leftrightarrow$ meine rote Krawatte
\end{document}










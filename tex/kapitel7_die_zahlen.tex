\documentclass[main.tex]{subfiles}

\begin{document}

\chapter{Die Zahlen}
\section{Kardinalzahlen}
\begin{minipage}{0,5\textwidth}
\subsection{0-19}
\begin{tabular}{ccc}

null &  &  \\ 

eins & $\rightarrow$ & elf \\ 
 
zwei & $\rightarrow$ & zwölf \\ 

drei & $\rightarrow$ & dreizehn \\ 

vier & $\rightarrow$ & vierzehn \\ 

fünf & $\rightarrow$ & fünfzehn \\ 

sechs & $\rightarrow$ & se\textcolor{red}{chz}ehn \\ 

sieben & $\rightarrow$ & sie\textcolor{red}{bz}ehn \\ 

acht & $\rightarrow$ & achtzehn \\ 

neun & $\rightarrow$ & neunzehn \\ 
 
zehn &  &  \\ 
 
\end{tabular} 

\end{minipage}



\begin{minipage}{0,5\textwidth}
\subsection{Zehner}
\begin{tabular}{ccc}
zehn &  &  \\ 
zwanzig & $\rightarrow$ & einundzwanzig \\ 
drießig & $\rightarrow$ & zweiunddreißig \\ 
vierzig & $\rightarrow$ & dreiundvierzig \\ 
fünfzig & $\rightarrow$ & sechsundfünfzig \\ 
se\textcolor{red}{chz}ig & $\rightarrow$ & siebenundsechzig \\ 
sie\textcolor{red}{bz}ig & $\rightarrow$ & achtundsiebzig \\ 
achtzig & $\rightarrow$ & neunundachtzig \\ 

neunzig & $\rightarrow$ & neunundneunzig \\ 

\end{tabular} 
\end{minipage}
\subsection{Hundert und großer}
(ein)hundert(und)eins\\
(ein)tausend(und)eins\\
eine Million, zwei Millionen\\
eine Milliarde, zwei Milliarden
\section{Ordnungszahlen}
Rangtelwoorden worden verbogen als adjectieven.
\subsection{Vorming}
$seq$ 19 + t + uitgang\\
$>$ 19 + st + uitgang\\
\subsection{Uitzondering}
erst/dritte/siebt/acht
\subsection{Afkorting}
10. $\rightarrow$ zehnt

\section{Bruchzahlen}

\end{document}